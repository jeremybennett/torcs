\documentclass[a4paper]{report}

%% Pre-amble - commonly defined macros.

%% Packages

\usepackage{enumerate}
\usepackage{amsmath}
\usepackage{amsfonts}
\usepackage{amssymb}
\usepackage{amsbsy}
\usepackage{theorem}
\usepackage{algorithm}
\usepackage{algorithmicx}
\usepackage{algpseudocode}
\usepackage{mathrsfs}
\usepackage{paralist}
\usepackage{epsfig}
\usepackage{subfigure}
\usepackage{makeidx} 
\usepackage{color}
\usepackage{pstricks}
\usepackage{pst-node}
\usepackage{multido} 
\usepackage{varioref}
\usepackage{wrapfig} %% for wrapping text around figures
%\usepackage{setspace} %% for using halfspaces
\usepackage[numbers]{natbib}
%\usepackage[left=2.5cm,right=2.5cm]{geometry}
%\usepackage[french, english]{babel}

\numberwithin{equation}{section} 

\theoremstyle{plain} \newtheorem{remark}{Remark}[section]
\theoremstyle{plain} \newtheorem{definition}{Definition}[section]
\theoremstyle{plain} \newtheorem{example}{Example}[section]
\theoremstyle{plain} \newtheorem{theorem}{Theorem}[section]
\theoremstyle{plain} \newtheorem{lemma}{Lemma}[section]
\theoremstyle{plain} \newtheorem{assumption}{Assumption}[section]



\newenvironment{proof}[1][Proof]{\begin{trivlist}
\item[\hskip \labelsep {\bfseries #1}]}{\end{trivlist}}
\newcommand{\qed}{\nobreak \ifvmode \relax \else
      \ifdim\lastskip<1.5em \hskip-\lastskip
      \hskip1.5em plus0em minus0.5em \fi \nobreak
      \vrule height0.5em width0.5em depth0.25em\fi}

\newcommand \E {\mathop{\mbox{\bf{E}}}\nolimits}
\newcommand \hE {\hat{\mathop{\mbox{\bf{E}}}\nolimits}}
\renewcommand \Pr {\mathop{\mbox{\bf{P}}}\nolimits}
\newcommand \given {\mathrel{|}}
\newcommand{\set}[2]{\{\, #1 \,:\, #2 \,\} }

%% Special characters

\newcommand \Reals {{\mathbb{R}}}
\newcommand \Naturals {{\mathbb{N}}}

\newcommand \FB {{\mathfrak{B}}}
\newcommand \FD {{\mathfrak{D}}}
\newcommand \FF {{\mathfrak{F}}}
\newcommand \FK {{\mathfrak{K}}}
\newcommand \FJ {{\mathfrak{J}}}
\newcommand \FL {{\mathfrak{L}}}
\newcommand \FO {{\mathfrak{O}}}
\newcommand \FS {{\mathfrak{S}}}
\newcommand \FT {{\mathfrak{T}}}
\newcommand \FP {{\mathfrak{P}}}
\newcommand \FR {{\mathfrak{R}}}


\newcommand \CA {{\mathcal{A}}}
\newcommand \CB {{\mathcal{B}}}
\newcommand \CC {{\mathcal{C}}}
\newcommand \CD {{\mathcal{D}}}
\newcommand \CE {{\mathcal{E}}}
\newcommand \CF {{\mathcal{F}}}
\newcommand \CG {{\mathcal{G}}}
\newcommand \CH {{\mathcal{H}}}
\newcommand \CJ {{\mathcal{J}}}
\newcommand \CL {{\mathcal{L}}}
\newcommand \CM {{\mathcal{M}}}
\newcommand \CN {{\mathcal{N}}}
\newcommand \CO {{\mathcal{O}}}
\newcommand \CP {{\mathcal{P}}}
\newcommand \CQ {{\mathcal{Q}}}
\newcommand \CR {{\mathcal{R}}}
\newcommand \CS {{\mathcal{S}}}
\newcommand \CT {{\mathcal{T}}}
\newcommand \CU {{\mathcal{U}}}
\newcommand \CV {{\mathcal{V}}}
\newcommand \CW {{\mathcal{W}}}
\newcommand \CX {{\mathcal{X}}}
\newcommand \CY {{\mathcal{Y}}}
\newcommand \CZ {{\mathcal{Z}}}

\newcommand \BA {{\mathbb{A}}}
\newcommand \BI {{\mathbb{I}}}
\newcommand \BS {{\mathbb{S}}}

\newcommand \bx {{\mathbf{x}}}
\newcommand \by {{\mathbf{y}}}
\newcommand \bu {{\mathbf{u}}}
\newcommand \bw {{\mathbf{w}}}
\newcommand \ba {{\mathbf{a}}}
\newcommand \bh {{\mathbf{h}}}
\newcommand \bo {{\mathbf{o}}}
\newcommand \bp {{\mathbf{p}}}
\newcommand \bs {{\mathbf{s}}}
\newcommand \br {{\mathbf{r}}}

\newcommand \SA {\mathscr{A}}
\newcommand \SB {\mathscr{B}}
\newcommand \SC {\mathscr{C}}
\newcommand \SF {\mathscr{F}}
\newcommand \SG {\mathscr{G}}
\newcommand \SH {\mathscr{H}}
\newcommand \SJ {\mathscr{J}}
\newcommand \SL {\mathscr{L}}
\newcommand \SP {\mathscr{P}}
%%\newcommand \SS {\mathscr{S}}
\newcommand \ST {\mathscr{T}}
\newcommand \SU {\mathscr{U}}
\newcommand \SV {\mathscr{V}}
\newcommand \SW {\mathscr{W}}


\newcommand \p {\partial}
\newcommand \D {\Delta}
%%\newcommand \d {\delta}

\newcommand \then{\Rightarrow}
\newcommand \defn {\mathrel{\triangleq}}
%\newcommand \StateSet {{\CQ}}


%% Commands

\newcommand \argmax{\mathop{\rm arg\,max}}
\newcommand \argmin{\mathop{\rm arg\,min}}
\newcommand \dtan{\mathop{\rm dtan}}
\newcommand \sgn{\mathop{\rm sgn}}
\newcommand \trace{\mathop{\rm trace}}

\newcommand \pnorm[2]{\|#1\|_{#2}}
\newcommand \inftynorm[1]{\|#1\|_\infty}
\newcommand \norm[1]{\|#1\|}


\newcommand \Lceil {\left\lceil}
\newcommand \Rceil {\right\rceil}


\newcommand\ind[1]{\mathop{\mbox{\ensuremath{\mathbb{I}}}}\left\{#1\right\}}
\newcommand\Ind{\mbox{\bf{I}}}

%%% Local Variables: 
%%% mode: latex
%%% TeX-master: "simuv3"
%%% End: 


\begin{document}

\chapter{Introduction}

\chapter{The message passing}

\chapter{The components}

\section{The differential}

A differential links together three components, $C = \{c_0, c_1,
c_2\}$.  By default, we shall refer to $c_0$ as the {\em input}
component and the other two components as the output components.  The
variables of interest in this setting are $\omega_i$, the angular
velocity of each of the coupled components, and $N_i$, the torque
applied by each component to the differential. Given the vectors
$\omega, N$, the differential computes a new torque vector $N'$ for
all of the components.

\subsection{Free differential}

A free, or open, differential links the two weels together through a
spider gear.

The reaction from each wheel is transmitted back to the spider
gear. If both wheels react equally, then the spider gear does not
turn. If one of the wheel is immobile, so that $\frac{1}{2} N_0 = N_1$
for example, then the reaction does not act against the drivetrain,
but since the spider gear can turn freely, it acts on the other wheel.
 
This system is equivalent to a rotating gear attached in between two
parallel surfaces, with the drive torque $N_0$ being equivalent to a
force acting in the center of the gear. If one surface is fixed, only
the other surface moves and all the force is 'transferred' to the
moving surface. Or, the way I like to think of it, the immobile
surface reacts with an equal and opposite force\footnote{For an object
  to remain at rest, all forces acting on it must sum to 0.} that
cancels $\frac{1}{2}N_0$ exactly and which is transmitted directly
with the rotating gear to the other, free, surface.

The output of this system looks as follows
\begin{align}
  N_s &= N_2 - N_1\\
  N'_1 &= \frac{1}{2} N_0 + N_s\\
  N'_2 &= \frac{1}{2} N_0 - N_s\\
  N'_0 &= - N_0.
\end{align}
			
\subsection{Viscous coupler}

Here we have a normal open differential, but we also pass torque
between the two components.  First, we define the current coupling
between the components, which depends on their relative angular velocity,
\begin{equation}
  \alpha \defn 1.0 - e^{-|\omega_1 - \omega_2|}.
\end{equation}
Then, we define the amount of bias, 
\begin{align}
  \beta &= \frac{1}{2}[1 + \alpha\sgn(\omega_2 - \omega_1)].
\end{align}
The bias is in addition limited to $\beta \in [\beta_{\min},
\beta_{\max}]$. We again set the spider gear torque to
\begin{align}
  N_s &= N_2 - N_1,
\end{align}
and in adition define the frictive torque
\begin{align}
  N_f &= \alpha v (\omega_2 - \omega_1).
\end{align}
Thus, the amount of coupling is determined by $\alpha$, which takes
the form of a friction coeffiient. The static friction is handled by
biasing the input torque, while the dynamic friction is handled via
the frictive torque $N_f$. When the system is totally uncoupled, the
differential behaves like an open differential.
The output torques are in fact
\begin{align}
  N'_1 &= \beta N_0 + N_s + N_f\\
  N'_2 &= (1 - \beta) N_0 - N_s - N_f\\
  N'_0 &= - N_0.
\end{align}
\end{document}

%%% Local Variables: 
%%% mode: latex
%%% TeX-master: t
%%% End: 
